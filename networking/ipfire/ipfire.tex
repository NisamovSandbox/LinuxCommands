\documentclass[12pt,a4paper]{article}
\usepackage[utf8]{inputenc}
\usepackage[spanish]{babel}
\usepackage{geometry}
\geometry{margin=2.5cm}
\usepackage{hyperref}
\usepackage{graphicx}
\usepackage{tikz}
\usetikzlibrary{shapes,arrows,positioning}

\title{Documentación de IPFire}
\author{}
\date{\today}

\begin{document}

\maketitle
\tableofcontents
\newpage

\section{Introducción}
IPFire es una \textbf{distribución Linux especializada en seguridad de redes} que funciona como \textbf{firewall}, router y sistema de control de tráfico.  
Su objetivo principal es proteger la red interna frente a accesos no autorizados, separando el tráfico en distintas \textbf{zonas de seguridad}.

\section{Zonas de red en IPFire}

IPFire organiza la red en diferentes colores, cada uno con un nivel de confianza distinto:

\subsection{RED}
\begin{itemize}
    \item Zona no confiable, normalmente conectada a Internet o a redes externas.
    \item Todo el tráfico entrante está bloqueado por defecto.
    \item Es la puerta hacia el exterior.
\end{itemize}

\subsection{GREEN}
\begin{itemize}
    \item Red interna segura (LAN).
    \item Dispositivos confiables como PCs, impresoras o NAS.
    \item Puede acceder a Internet a través de IPFire.
\end{itemize}

\subsection{ORANGE}
\begin{itemize}
    \item Zona semi-segura o DMZ (Desmilitarizada).
    \item Aloja servidores públicos como web, FTP o correo.
    \item Separada de GREEN para proteger la red interna en caso de compromiso.
\end{itemize}

\section{Instalación básica}
\begin{enumerate}
    \item Arrancar desde la ISO de IPFire.
    \item Seleccionar idioma, teclado y zona horaria.
    \item Configurar contraseñas:
    \begin{itemize}
        \item \texttt{root}: acceso por consola
        \item \texttt{admin}: acceso a la interfaz web
    \end{itemize}
\end{enumerate}

\section{Configuración de red}

\subsection{Acceso al modo configuración}
Desde la consola de IPFire, se ingresa al menú de configuración con:

\begin{verbatim}
setup
\end{verbatim}

\subsection{Asignación de interfaces de red}
Las interfaces se asignan según la \textbf{dirección MAC} de cada adaptador:

\begin{itemize}
    \item GREEN: adaptador de red interna
    \item RED: adaptador conectado a Internet
    \item ORANGE: adaptador para la DMZ
\end{itemize}

\subsection{Configuración de IP por zona}
\begin{itemize}
    \item GREEN: IP estática (ej: 192.168.1.1), máscara 255.255.255.0
    \item RED: DHCP (o IP estática si el ISP lo requiere)
    \item ORANGE: IP estática (ej: 192.168.2.1)
\end{itemize}

\section{Acceso a la interfaz web}
Desde un equipo en GREEN:

\begin{verbatim}
https://IP_GREEN:444
Usuario: admin
\end{verbatim}

\section{Políticas de firewall por defecto}
\begin{itemize}
    \item GREEN puede acceder a Internet
    \item RED no puede acceder a la red interna
    \item ORANGE tiene acceso controlado
    \item Todo lo no permitido está bloqueado por defecto
\end{itemize}

\section{Servicios disponibles}
\begin{itemize}
    \item DHCP
    \item DNS
    \item Proxy web
    \item IDS/IPS (Snort o Suricata)
    \item VPN (OpenVPN, IPsec)
\end{itemize}

\section{Ejemplo de laboratorio}
\begin{itemize}
    \item IPFire instalado en una máquina virtual
    \item Tres adaptadores de red:
    \begin{itemize}
        \item Bridge: RED
        \item Red interna: GREEN
        \item Red interna: ORANGE
    \end{itemize}
    \item Clientes conectados a GREEN
    \item Servidor web ubicado en ORANGE
\end{itemize}

\section{Comandos útiles}
\begin{verbatim}
setup        # configuración del sistema
reboot       # reiniciar
halt         # apagar
ifconfig     # ver interfaces de red
ip a         # mostrar direcciones IP
\end{verbatim}

\end{document}
