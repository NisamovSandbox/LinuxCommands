\documentclass[12pt,a4paper]{article}

% Codificación y lenguaje
\usepackage[spanish]{babel}
\usepackage[utf8]{inputenc}
\usepackage[T1]{fontenc}

% Márgenes
\usepackage{geometry}
\geometry{margin=2.5cm}

% Paquetes útiles
\usepackage{hyperref}
\usepackage{xcolor}
\usepackage{listings}
\usepackage{setspace}

% Configuración listings
\lstset{
  basicstyle=\ttfamily\small,
  frame=single,
  breaklines=true,
  keywordstyle=\color{blue},
  commentstyle=\color{gray},
  stringstyle=\color{teal},
  showstringspaces=false
}

\title{Documentación Apache2}
\author{Andrés Ruslan Abadías Otal (Nisamov)}
\date{\today}

\begin{document}

\maketitle
\onehalfspacing

\section{Instalación}

Para evitar posibles problemas durante la instalación, se recomienda actualizar los paquetes del sistema.

\begin{lstlisting}[language=bash]
# Esto es únicamente opcional
sudo apt update && sudo apt upgrade -y
\end{lstlisting}

Instalación de Apache2 con privilegios de administrador:

\begin{lstlisting}[language=bash]
sudo apt install apache2
\end{lstlisting}

\section{Creación de la página}

Para crear una página hay que dirigirse a la ruta \texttt{/var/www/}, donde se encuentra el fichero \texttt{index.html}.
Se debe crear un directorio con el nombre del dominio deseado, finalizando en \texttt{.es}, \texttt{.org}, \texttt{.com}, etc.

\begin{lstlisting}[language=bash]
cd /var/www/
mkdir mipagina.es
cd mipagina.es
touch index.html
\end{lstlisting}

\section{Edición de la página}

Para dar estructura a la página, es necesario editar el fichero \texttt{index.html}.  
Si se desea añadir diseño y funcionalidad, se utilizarán ficheros \texttt{.css} y \texttt{.js}.

\begin{lstlisting}[language=bash]
sudo nano index.html
\end{lstlisting}

Contenido básico HTML5:

\begin{lstlisting}[language=html]
<!DOCTYPE html>
<html lang="es">
<head>
    <meta charset="UTF-8">
    <meta name="viewport" content="width=device-width, initial-scale=1.0">
    <title>MiPagina</title>
    <link href="styles/style.css" rel="stylesheet" type="text/css" />
</head>
<body>
    <p>Este es el contenido de tu pagina</p>
    <u>He seguido la guia de github.com/Nisamov</u>
    <script src="scripts/script.js"></script>
</body>
</html>
\end{lstlisting}

\section{Estructura de directorios}

La estructura recomendada es la siguiente:

\begin{lstlisting}
var
└── www
    └── mipagina.es
        ├── index.html
        ├── styles
        │   └── style.css
        └── scripts
            └── script.js
\end{lstlisting}

Creación de los directorios:

\begin{lstlisting}[language=bash]
cd /var/www/mipagina.es
mkdir styles scripts
\end{lstlisting}

\section{Estilo y funcionamiento con CSS y JavaScript}

\subsection{JavaScript}

\begin{lstlisting}[language=bash]
cd /var/www/mipagina.es/scripts
nano script.js
\end{lstlisting}

Contenido de ejemplo:

\begin{lstlisting}[language=JavaScript]
// Alerta de comprobación
alert("El Codigo JavaScript Funciona correctamente");
console.log("Correcto funcionamiento");
\end{lstlisting}

\subsection{CSS}

\begin{lstlisting}[language=bash]
cd /var/www/mipagina.es/styles
nano style.css
\end{lstlisting}

Ejemplo de estilos:

\begin{lstlisting}[language=CSS]
body {
  background-color: #fefbd8;
}

h1 {
  background-color: #80ced6;
}

p {
  font-family: 'Open Sans';
  font-size: 14px;
  color: #ccc;
  line-height: 18px;
}
\end{lstlisting}

\section{Configuración SSL}

Instalación de OpenSSL:

\begin{lstlisting}[language=bash]
sudo apt install openssl
\end{lstlisting}

Habilitar módulo SSL y reiniciar Apache:

\begin{lstlisting}[language=bash]
sudo a2enmod ssl
sudo systemctl restart apache2
\end{lstlisting}

Clonar el fichero de configuración:

\begin{lstlisting}[language=bash]
cd /etc/apache2/sites-available
sudo cp 000-default.conf mipagina.es.conf
\end{lstlisting}

Configuración del VirtualHost:

\begin{lstlisting}[language=bash]
<VirtualHost *:80>
ServerAdmin webmaster@localhost
DocumentRoot /var/www/mipagina.es
Redirect permanent / http://www.mipagina.es
ServerName www.mipagina.es
</VirtualHost>

<VirtualHost *:443>
ServerName www.mipagina.es
DocumentRoot /var/www/mipagina.es
SSLEngine on
SSLCertificateFile /etc/apache2/certificate/apache-certificado.crt
SSLCertificateKeyFile /etc/apache2/certificate/apache.key
</VirtualHost>
\end{lstlisting}

Activar el sitio y reiniciar Apache:

\begin{lstlisting}[language=bash]
sudo a2ensite mipagina.es
sudo systemctl restart apache2
\end{lstlisting}

\section{Configuración de hosts}

\begin{lstlisting}
127.0.0.1       localhost
127.0.1.1       nisamov
40.0.0.2        mipagina.es www.mipagina.es
\end{lstlisting}

\section{Conclusión}

Si se ha seguido correctamente el procedimiento descrito, la página será completamente funcional.
Durante el proceso pueden aparecer errores, los cuales se mostrarán en los reinicios y registros del servicio Apache2.

\end{document}